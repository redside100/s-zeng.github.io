%%%%%%%%%%%%%%%%%
% This is an sample CV template created using altacv.cls
% (v1.1.4, 27 July 2018) written by LianTze Lim (liantze@gmail.com). Now compiles with pdfLaTeX, XeLaTeX and LuaLaTeX.
%
%% It may be distributed and/or modified under the
%% conditions of the LaTeX Project Public License, either version 1.3
%% of this license or (at your option) any later version.
%% The latest version of this license is in
%%    http://www.latex-project.org/lppl.txt
%% and version 1.3 or later is part of all distributions of LaTeX
%% version 2003/12/01 or later.
%%%%%%%%%%%%%%%%

%% If you need to pass whatever options to xcolor
\PassOptionsToPackage{dvipsnames}{xcolor}

%% If you are using \orcid or academicons
%% icons, make sure you have the academicons
%% option here, and compile with XeLaTeX
%% or LuaLaTeX.
% \documentclass[10pt,a4paper,academicons]{altacv}

%% Use the "normalphoto" option if you want a normal photo instead of cropped to a circle
% \documentclass[10pt,a4paper,normalphoto]{altacv}

\documentclass[10pt,letterpaper]{altacv}

%% AltaCV uses the fontawesome and academicon fonts
%% and packages.
%% See texdoc.net/pkg/fontawecome and http://texdoc.net/pkg/academicons for full list of symbols.
%%
%% Compile with LuaLaTeX for best results. If you
%% want to use XeLaTeX, you may need to install
%% Academicons.ttf in your operating system's font
%% folder.


% Change the page layout if you need to
\geometry{left=1cm,right=9cm,marginparwidth=7.0cm,marginparsep=1.2cm,top=1.20cm,bottom=1.25cm}

% Change the font if you want to.

% If using pdflatex:
\usepackage[utf8]{inputenc}
\usepackage[T1]{fontenc}
% \usepackage[default]{lato}

% If using xelatex or lualatex:
\setmainfont{Lato}

% Change the colours if you want to
\definecolor{Mulberry}{HTML}{92243D}
\definecolor{Cool}{HTML}{E2243D}
\definecolor{SlateGrey}{HTML}{2E2E2E}
\definecolor{LightGrey}{HTML}{666666}
\colorlet{heading}{Cool}
\colorlet{accent}{Mulberry}
\colorlet{emphasis}{SlateGrey}
\colorlet{body}{LightGrey}

% Change the bullets for itemize and rating marker
% for \cvskill if you want to
\renewcommand{\itemmarker}{{\small\textbullet}}
\renewcommand{\ratingmarker}{\faCircle}

%% sample.bib contains your publications
\addbibresource{sample.bib}

\begin{document}
\name{Simon Zeng}
\tagline{Software Developer}
% \photo{2.8cm}{pic2}
\personalinfo{%
  % Not all of these are required!
  % You can add your own with \printinfo{symbol}{detail}
  \email{simon.zeng@uwaterloo.ca}
  \phone{613-983-9079}
  % \mailaddress{58 Akenhead Crescent, Kanata, ON, K2T0B4}
  \homepage{s-zeng.github.io}
  % \twitter{@twitterhandle}
  \linkedin{linkedin.com/in/s-zeng1}
  \github{github.com/s-zeng}
  % \printinfo{UW}{20769883}
  %% You MUST add the academicons option to \documentclass, then compile with LuaLaTeX or XeLaTeX, if you want to use \orcid or other academicons commands.
%   \orcid{orcid.org/0000-0000-0000-0000}
}

%% Make the header extend all the way to the right, if you want.
\begin{fullwidth}
\makecvheader
\end{fullwidth}

%% Depending on your tastes, you may want to make fonts of itemize environments slightly smaller
% \AtBeginEnvironment{itemize}{\small}

%% Provide the file name containing the sidebar contents as an optional parameter to \cvsection.
%% You can always just use \marginpar{...} if you do
%% not need to align the top of the contents to any
%% \cvsection title in the "main" bar.
\cvsection[sidebar]{Experience}
% \cvsection{Experience}

\cvevent{Performance Engineering/Test Automation Co-op}{CENX}{July 2017 -- September 2017}{Ottawa, Ontario}
% Developed low-level network interface technologies in order to maximize artificial 
% demand/stress on company products for the purpose of augmenting performance, 
% assuring quality, and detecting security issues.
\begin{itemize}
    \item Automated stress testing on company products and networks using 
        \textbf{Python}
    \item Developed custom TCP-interfacing implementations of IETF RFCs in 
        \textbf{Bash} and \textbf{Python} to properly debug non 
        standards-compliant network stacks
    \item Caught and reported multiple bugs and issues in \textbf{Clojure} code, 
        leading to large performance increases and major security fixes
\end{itemize}

\divider

\cvevent{Full Stack Web Development Intern}{inBay Technologies}{July 2016 -- August 2016}{Kanata, Ontario}
% Created internal-use development tools, performing both front-end and back-end 
% development using web technologies such as Ruby on Rails and node.js
\begin{itemize}
    \item Fetched and formatted statistics and metrics from network endpoints 
        using \textbf{Ruby (on Rails)} and \textbf{Sinatra}
    \item Constructed front-end web pages using \textbf{HTML}, \textbf{CSS}, and 
        \textbf{Coffeescript} to present statistics to clients
\end{itemize}

% \divider

% \cvevent{Volunteer STEM Camp Counsellor}{Virtual Ventures}{July 2015 -- August 2015}{Ottawa, Ontario}
% Taught lessons and lead activities focusing on programming, robotics, and 
% physics for children aged 9-12

\bigskip

\cvsection{Projects}

\cvevent{Polynomial Interpolator}{github.com/s-zeng/interpoly}{}{}
\begin{itemize}
    \item Developed a small \textbf{Haskell} based CLI utility that generates a 
        minimum degree polynomial to interpolate over any finite list of numbers
\end{itemize}

\divider

\cvevent{Neural Network Based Automatic Midi Music Generator}{github.com/s-zeng/rag-shenanigann}{}{}
\begin{itemize}
    \item Scraped and preprocessed web MIDI files into neural network 
        consumable text format with a suite of \textbf{Python} scripts
    \item Automated training deployment of \textbf{Lua} based recurrent neural 
        networks with \textbf{Docker}
\end{itemize}

\divider

\cvevent{Discord API Music Player Bot}{github.com/s-zeng/Zengyatta}{}{}
\begin{itemize}
    \item Interfaced with the Discord REST API using \textbf{Java} to create a 
        bot that plays music in voice channels
    \item Reverse engineered Bandcamp's web services to enable the use of 
        Bandcamp links as a music source
\end{itemize}


% \cvevent{repl.vim}{github.com/ujihisa/repl.vim}{}{}
% Vim plugin that provides bindings for summoning an interactive environment with 
% the code that is being written. My contributions include additional code to make 
% the plugin compatible with Racket/Scheme source files and interactive environments

\medskip

% \cvsection{A Day of My Life}

% Adapted from @Jake's answer from http://tex.stackexchange.com/a/82729/226
% \wheelchart{outer radius}{inner radius}{
% comma-separated list of value/text width/color/detail}
% \wheelchart{1.5cm}{0.5cm}{%
  % 6/8em/accent!30/{Sleep,\\beautiful sleep},
  % 3/8em/accent!40/Hopeful novelist by night,
  % 8/8em/accent!60/Daytime job,
  % 2/10em/accent/Sports and relaxation,
  % 5/6em/accent!20/Spending time with family
% }

% \clearpage
% \cvsection[page2sidebar]{Publications}

% \nocite{*}

% \printbibliography[heading=pubtype,title={\printinfo{\faBook}{Books}},type=book]

% \divider

% \printbibliography[heading=pubtype,title={\printinfo{\faFileTextO}{Journal Articles}},type=article]

% \divider

% \printbibliography[heading=pubtype,title={\printinfo{\faGroup}{Conference Proceedings}},type=inproceedings]

%% If the NEXT page doesn't start with a \cvsection but you'd
%% still like to add a sidebar, then use this command on THIS
%% page to add it. The optional argument lets you pull up the
%% sidebar a bit so that it looks aligned with the top of the
%% main column.
% \addnextpagesidebar[-1ex]{page3sidebar}


\end{document}
